\begin{table}[htbp]\centering
\begin{footnotesize}
\def\sym#1{\ifmmode^{#1}\else\(^{#1}\)\fi}
\caption{Caracter�sticas populacionais e socioecon�micas do munic�pios \label{tab:varcontroles}}
\begin{tabular}{l*{3}{c}}
\toprule
          \multicolumn{1}{c}{(1)}&\multicolumn{1}{c}{(2)}&\multicolumn{1}{c}{(3)}  \\
          \multicolumn{1}{c}{Vari�veis}&\multicolumn{1}{c}{M�dia}&\multicolumn{1}{c}{Desvio padr�o} \\
\midrule
Taxa de municipaliza��o &    75,6\% &    20,1\%   \\
%\addlinespace
%FUNDEF/gastos &    43,1 &    91,3  \\
%\addlinespace
Popula��o de 07 a 14 anos &    5.108\ &    25.585   \\
%\addlinespace
Popula��o feminina de 25 a 64 anos &     7.195   &    49.452 \\
%\addlinespace
Popula��o com mais de 65 anos &      1825  &  12.707  \\
%\addlinespace
Popula��o  &              31.577  &  189.558  \\
%\addlinespace
Vacinas &                    12.210  &  66.960 \\
%\addlinespace
Mortalidade infantil &   20,1  &  21,6 \\
%\addlinespace
PIB \emph{per capita} (R\$1.000,00) &   4,50  &  5,23  \\
%\addlinespace
\midrule
\(N\)     &   2.837        &   2.837          \\
\bottomrule
\multicolumn{3}{l}{\footnotesize Fonte: Calculos pr�prios a partir de dados do Censo Escolar, DataSUS e IPEAdata} \\
\end{tabular}
\end{footnotesize}
\end{table}
