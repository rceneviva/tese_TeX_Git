\begin{table}[htbp]\centering
\begin{footnotesize}
\def\sym#1{\ifmmode^{#1}\else\(^{#1}\)\fi}
\caption{Distribui��o dos grupos tratamento e controle  
\newline segundo as regi�es do Brasil \label{tab:regiao}}
\begin{tabular}{l*{4}{c}}
\toprule
              \bf{Regi�o} &  \bf{Grupo EE}& \bf{Grupo MM} & \bf{Grupo EM} \\
              \bf{geogr�fica} & \bf{(Controle I)} & \bf{(Controle II)} &\bf{(Tratamento)} \\
               
\midrule
Norte &   24,38\% &    18,81\%&    09,56\%   \\
%\addlinespace
Nordeste &  31,74\% &    40,61\%&    41,07\%\\
%\addlinespace
Sudeste & 14,27\% &    15,65\%&    29,52\%  \\
%\addlinespace
Sul  &  15,01\% &    15,33\%&    17,15\% \\
%\addlinespace
Centro Oeste & 14,60\% &    09,61\%&    02,70\% \\
%\addlinespace
\midrule
\bottomrule
\multicolumn{4}{l}{\footnotesize Fonte: Calculos pr�prios a partir de dados do SAEB/INEP}\\
\multicolumn{4}{l}{\footnotesize Notas: EE: escolas estaduais que permaneceram estaduais }\\
\multicolumn{4}{l}{\footnotesize  MM: escolas municipais que permaneceram municipais}\\
\multicolumn{4}{l}{\footnotesize  EM: escolas estaduais que migraram para as redes municipais}\\
\end{tabular}
\end{footnotesize}
\end{table}
