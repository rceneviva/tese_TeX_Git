\begin{table}[htbp]\centering
\begin{footnotesize}
\def\sym#1{\ifmmode^{#1}\else\(^{#1}\)\fi}
\caption{Profici�ncia SAEB, 4a. s�rie do ensino fundamental   \newline  L�ngua Portuguesa \label{tab:profic_port}}
\begin{tabular}{l*{6}{c}}
\toprule
          &\multicolumn{1}{c}{(1)}&\multicolumn{1}{c}{(2)}&\multicolumn{1}{c}{(3)}&\multicolumn{1}{c}{(4)}&\multicolumn{1}{c}{(5)} &\multicolumn{1}{c}{(6)}\\
          &\multicolumn{1}{c}{SAEB}&\multicolumn{1}{c}{SAEB}&\multicolumn{1}{c}{SAEB}&\multicolumn{1}{c}{Painel}&\multicolumn{1}{c}{Painel} &\multicolumn{1}{c}{Painel}\\
          &\multicolumn{1}{c}{Privadas}&\multicolumn{1}{c}{Estaduais}&\multicolumn{1}{c}{Municipais}&\multicolumn{1}{c}{Privadas}&\multicolumn{1}{c}{Estaduais} &\multicolumn{1}{c}{Municipais}\\

\midrule
1997 &   215,33 &    174,68 &    170,58& 217,64 &    173,71&    168,92  \\
%\addlinespace
1999 &   205,90   &    161,62&    157,34& 208,17   &    161,82&    158,70 \\
%\addlinespa
2001 &   204,72  &  159,35&   153,25&  206,79  &  159,06&   152,50 \\
%\addlinespace
2003 &      212,72&    166,38&   162,96&    212,93&    165,42& 161,17   \\
%\addlinespace
2005 &      209,74&    167,09&    162,85&     211,14&    168,27&    163,54  \\
%\addlinespace

\midrule
\(N\)     &   45.568         &   50.785        &   56.662         &   29.787  & 32.381    & 34.546      \\
\bottomrule
\multicolumn{6}{l}{\footnotesize Fonte: Calculos pr�prios a partir de dados do SAEB/INEP}\\
\end{tabular}
\end{footnotesize}
\end{table}
